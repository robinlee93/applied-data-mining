

\title{HW5 For Applied Data Mining \\ STAT W 3026-4026 \\ Spring 2016 \\ Columbia University}
\author{
       Robin Lee  \\
               rcl2136\\
        QMSS MA}
\date{\today}

\documentclass[12pt]{article}

\makeatletter
\newcommand*{\rom}[1]{\expandafter\@slowromancap\romannumeral #1@}
\makeatother
\usepackage{amsthm,amsmath,graphicx,csquotes}
\usepackage[
         colorlinks=true,
         linkcolor=black,
         citecolor=black,
         urlcolor=black]
         {hyperref}  
         
\newtheorem{thm}{Theorem}
\newtheorem{defn}{Definition}
%\newtheorem{rem}{Remark}
%\newtheorem{lem}{Lemma}
\newtheorem{prop}{Property}


\usepackage{Sweave}
\begin{document}
\Sconcordance{concordance:Lee_Robin_hw5.tex:Lee_Robin_hw5.Rnw:%
1 29 1 1 0 71 1 1 2 1 0 2 1 2 2 1 0 2 1 3 0 1 2 1 1 1 5 7 0 1 2 1 1 1 4 %
3 0 1 1 1 2 1 1 3 0 1 2 1 1 1 4 3 0 1 1 1 2 1 1 3 0 1 2 1 1 1 4 3 0 1 1 %
1 2 1 1 3 0 1 2 1 1 1 4 3 0 1 1 1 2 1 1 3 0 1 2 1 1 1 4 3 0 1 1 1 2 1 1 %
3 0 1 2 2 1 1 4 3 0 1 1 22 0 1 2 1 1 7 0 1 2 1 1 1 4 3 0 1 1 1 2 1 1 3 %
0 1 2 1 1 1 4 3 0 1 1 1 2 1 1 3 0 1 2 2 1 1 3 1 0 1 1 10 0 1 2 3 1 16 0 %
1 5}

\maketitle


\section{Instruction}
For the Boston Housing Data, firstpartition the data as training (2/3) and testing (1/3).
Then fit each the following models in table below (Linear Reg, Lasso, ElasticNet, PLS, Neural Networks, MARS, SVM, K-NN) and present the performance
measures for both training and test data. Use cross validation to tune the parameters. 

\newpage

\section{Result}
\begin{table}[ht]

\centering % used for centering table

\begin{tabular}{c c c c c} % centered columns (4 columns)

\hline %inserts horizontal lines

& & Training & Testing \\ [0.5ex] % inserts table

\hline %inserts horizontal lines

& RMSE & R2 & RMSE & R2 \\ [0.5ex] % inserts table

\hline % inserts single horizontal line

Linear Regression & 4.515 & 0.775 & 5.587  & 0.614 \\ [0.5ex] % inserts table

\hline %inserts horizontal lines

Lasso & 4.528 & 0.769 & 5.613 & 0.609\\ [0.5ex] % inserts table

\hline %inserts horizontal lines

ElasticNet &  4.531 & 0.776 &  5.586 & 0.613\\ [0.5ex] % inserts table
\hline %inserts horizontal lines

PLS & 7.434 & 0.374 & 7.704 & 0.272\\ [0.5ex] % inserts table
\hline %inserts horizontal lines

Neural Networks & 23.578 &  NA &  23.019 & NA \\ [0.5ex] % inserts table

\hline %inserts horizontal lines

MARS & 3.108 &  0.888 & 4.718 & 0.735\\ [0.5ex] % inserts table

\hline %inserts horizontal lines

SVM & 3.667 & 0.855 & 4.566 & 0.744\\ [0.5ex] % inserts table

\hline %inserts horizontal lines

K-NN &  6.369 & 0.547 & 6.749 & 0.446\\ [0.5ex] % inserts table

\hline %inserts horizontal lines

\end{tabular}

\label{table:nonlin} % is used to refer this table in the text

\end{table}






\section{Step 1 - Create Data Partition}

\begin{Schunk}
\begin{Sinput}
> library(caret)
> library(MASS)
> data("Boston")
> set.seed(569)
> train_index <- createDataPartition(Boston$medv, p = 2/3, 
+                                   list = FALSE, times = 1)
> train <- Boston[train_index, ] 
> test <- Boston[-train_index, ]
\end{Sinput}
\end{Schunk}

\section{Set CV}
\begin{Schunk}
\begin{Sinput}
> control <- trainControl( # 10 fold CV, repeated 10 times
+   method = 'repeatedcv', number = 10, repeats = 10
+   
+ )
\end{Sinput}
\end{Schunk}

\section{Linear Regression}
\begin{Schunk}
\begin{Sinput}
> lm1 <- train(medv ~ . , data = train, 
+             method = "lm", 
+             trControl = control)
> lm1
> lm_test <- predict(lm1, test)
> postResample(lm_test, test$medv)
\end{Sinput}
\end{Schunk}

\section{Lasso}
\begin{Schunk}
\begin{Sinput}
> lasso1 <- train(medv ~ . , data = train, 
+             method = "lasso", 
+             trControl = control)
> lasso1
> lasso_test <- predict(lasso1, test)
> postResample(lasso_test, test$medv)
\end{Sinput}
\end{Schunk}

\section{Elastic Net}
\begin{Schunk}
\begin{Sinput}
> enet1 <- train(medv ~ . , data = train, 
+             method = "enet", 
+             trControl = control)
> enet1
> enet_test <- predict(enet1, test)
> postResample(enet_test, test$medv)
\end{Sinput}
\end{Schunk}

\section{Partial Least Square}
\begin{Schunk}
\begin{Sinput}
> pls1 <- train(medv ~ . , data = train, 
+             method = "pls", 
+             trControl = control)
> pls1
> pls_test <- predict(pls1, test)
> postResample(pls_test, test$medv)
\end{Sinput}
\end{Schunk}

\section{Neural Net}
\begin{Schunk}
\begin{Sinput}
> nnet1 <- train(medv ~ . , data = train, 
+             method = "nnet", 
+             trControl = control)
> nnet1
> nnet_test <- predict(nnet1, test)
> postResample(nnet_test, test$medv)
\end{Sinput}
\end{Schunk}


\section{MARS}
\begin{Schunk}
\begin{Sinput}
> mars1 <- train(medv ~ . , data = train, 
+             method = "earth", 
+             trControl = control)
> mars1
\end{Sinput}
\begin{Soutput}
Multivariate Adaptive Regression Spline 

338 samples
 13 predictors

No pre-processing
Resampling: Cross-Validated (10 fold, repeated 10 times) 
Summary of sample sizes: 305, 304, 303, 305, 306, 304, ... 
Resampling results across tuning parameters:

  nprune  RMSE      Rsquared   RMSE SD    Rsquared SD
   2      5.757601  0.6157302  0.9035042  0.13191191 
  11      3.369326  0.8704629  0.5822574  0.04774102 
  20      3.109791  0.8895437  0.5068076  0.04044185 

Tuning parameter 'degree' was held constant at a value of 1
RMSE was used to select the optimal model using  the smallest value.
The final values used for the model were nprune = 20 and degree = 1. 
\end{Soutput}
\begin{Sinput}
> mars_test <- predict(mars1, test)
> postResample(mars_test, test$medv)
\end{Sinput}
\begin{Soutput}
     RMSE  Rsquared 
4.7180742 0.7359379 
\end{Soutput}
\end{Schunk}

\section{SVM}
\begin{Schunk}
\begin{Sinput}
> svm1 <- train(medv ~ . , data = train, 
+             method = "svmRadial", 
+             trControl = control)
> svm1
> svm_test <- predict(svm1, test)
> postResample(svm_test, test$medv)
\end{Sinput}
\end{Schunk}

\section{K Nearest Neighbor}
\begin{Schunk}
\begin{Sinput}
> knn1 <- train(medv ~ . , data = train, 
+             method = "knn", 
+             trControl = control)
> knn1
> knn_test <- predict(knn1, test)
> postResample(knn_test, test$medv)
\end{Sinput}
\end{Schunk}

\section{Question 2}
The lm.ridge function would not fit age twice. I created an age2 variable that is the same as age. Fitting it twice age with lm.ridge does not give the expected coefficient . It is not one half of the original coefficient. 
\begin{Schunk}
\begin{Sinput}
> ridgem <- lm.ridge(medv ~ age + age + . , data = train)
> ridgem
\end{Sinput}
\begin{Soutput}
                      age         crim           zn        indus         chas 
 19.48390148  -0.01368204  -0.09892445   0.03403269   0.03776729   2.82505346 
         nox           rm          dis          rad          tax      ptratio 
-13.88439925   5.31382481  -1.24210651   0.26774200  -0.01342864  -0.76619515 
       black        lstat 
  0.01223820  -0.39446523 
\end{Soutput}
\begin{Sinput}
> train_add <- train
> train_add$age2 <- train_add$age
> ridgem2 <- lm.ridge(medv ~ age +., data = train_add)
> ridgem2
\end{Sinput}
\begin{Soutput}
                        age          crim            zn         indus 
 2.435153e+01 -8.834433e+13 -1.223741e-01  6.734561e-03  1.340861e-01 
         chas           nox            rm           dis           rad 
 3.968855e+00 -1.671082e+01  5.280787e+00 -1.264933e+00  2.339035e-01 
          tax       ptratio         black         lstat          age2 
-1.342864e-02 -7.661951e-01  1.125007e-02 -3.944652e-01  8.834433e+13 
\end{Soutput}
\begin{Sinput}
> 
> 
> 
\end{Sinput}
\end{Schunk}
\end{document}
